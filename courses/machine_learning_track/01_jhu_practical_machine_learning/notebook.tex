
% Default to the notebook output style

    


% Inherit from the specified cell style.




    
\documentclass[11pt]{article}

    
    
    \usepackage[T1]{fontenc}
    % Nicer default font (+ math font) than Computer Modern for most use cases
    \usepackage{mathpazo}

    % Basic figure setup, for now with no caption control since it's done
    % automatically by Pandoc (which extracts ![](path) syntax from Markdown).
    \usepackage{graphicx}
    % We will generate all images so they have a width \maxwidth. This means
    % that they will get their normal width if they fit onto the page, but
    % are scaled down if they would overflow the margins.
    \makeatletter
    \def\maxwidth{\ifdim\Gin@nat@width>\linewidth\linewidth
    \else\Gin@nat@width\fi}
    \makeatother
    \let\Oldincludegraphics\includegraphics
    % Set max figure width to be 80% of text width, for now hardcoded.
    \renewcommand{\includegraphics}[1]{\Oldincludegraphics[width=.8\maxwidth]{#1}}
    % Ensure that by default, figures have no caption (until we provide a
    % proper Figure object with a Caption API and a way to capture that
    % in the conversion process - todo).
    \usepackage{caption}
    \DeclareCaptionLabelFormat{nolabel}{}
    \captionsetup{labelformat=nolabel}

    \usepackage{adjustbox} % Used to constrain images to a maximum size 
    \usepackage{xcolor} % Allow colors to be defined
    \usepackage{enumerate} % Needed for markdown enumerations to work
    \usepackage{geometry} % Used to adjust the document margins
    \usepackage{amsmath} % Equations
    \usepackage{amssymb} % Equations
    \usepackage{textcomp} % defines textquotesingle
    % Hack from http://tex.stackexchange.com/a/47451/13684:
    \AtBeginDocument{%
        \def\PYZsq{\textquotesingle}% Upright quotes in Pygmentized code
    }
    \usepackage{upquote} % Upright quotes for verbatim code
    \usepackage{eurosym} % defines \euro
    \usepackage[mathletters]{ucs} % Extended unicode (utf-8) support
    \usepackage[utf8x]{inputenc} % Allow utf-8 characters in the tex document
    \usepackage{fancyvrb} % verbatim replacement that allows latex
    \usepackage{grffile} % extends the file name processing of package graphics 
                         % to support a larger range 
    % The hyperref package gives us a pdf with properly built
    % internal navigation ('pdf bookmarks' for the table of contents,
    % internal cross-reference links, web links for URLs, etc.)
    \usepackage{hyperref}
    \usepackage{longtable} % longtable support required by pandoc >1.10
    \usepackage{booktabs}  % table support for pandoc > 1.12.2
    \usepackage[inline]{enumitem} % IRkernel/repr support (it uses the enumerate* environment)
    \usepackage[normalem]{ulem} % ulem is needed to support strikethroughs (\sout)
                                % normalem makes italics be italics, not underlines
    

    
    
    % Colors for the hyperref package
    \definecolor{urlcolor}{rgb}{0,.145,.698}
    \definecolor{linkcolor}{rgb}{.71,0.21,0.01}
    \definecolor{citecolor}{rgb}{.12,.54,.11}

    % ANSI colors
    \definecolor{ansi-black}{HTML}{3E424D}
    \definecolor{ansi-black-intense}{HTML}{282C36}
    \definecolor{ansi-red}{HTML}{E75C58}
    \definecolor{ansi-red-intense}{HTML}{B22B31}
    \definecolor{ansi-green}{HTML}{00A250}
    \definecolor{ansi-green-intense}{HTML}{007427}
    \definecolor{ansi-yellow}{HTML}{DDB62B}
    \definecolor{ansi-yellow-intense}{HTML}{B27D12}
    \definecolor{ansi-blue}{HTML}{208FFB}
    \definecolor{ansi-blue-intense}{HTML}{0065CA}
    \definecolor{ansi-magenta}{HTML}{D160C4}
    \definecolor{ansi-magenta-intense}{HTML}{A03196}
    \definecolor{ansi-cyan}{HTML}{60C6C8}
    \definecolor{ansi-cyan-intense}{HTML}{258F8F}
    \definecolor{ansi-white}{HTML}{C5C1B4}
    \definecolor{ansi-white-intense}{HTML}{A1A6B2}

    % commands and environments needed by pandoc snippets
    % extracted from the output of `pandoc -s`
    \providecommand{\tightlist}{%
      \setlength{\itemsep}{0pt}\setlength{\parskip}{0pt}}
    \DefineVerbatimEnvironment{Highlighting}{Verbatim}{commandchars=\\\{\}}
    % Add ',fontsize=\small' for more characters per line
    \newenvironment{Shaded}{}{}
    \newcommand{\KeywordTok}[1]{\textcolor[rgb]{0.00,0.44,0.13}{\textbf{{#1}}}}
    \newcommand{\DataTypeTok}[1]{\textcolor[rgb]{0.56,0.13,0.00}{{#1}}}
    \newcommand{\DecValTok}[1]{\textcolor[rgb]{0.25,0.63,0.44}{{#1}}}
    \newcommand{\BaseNTok}[1]{\textcolor[rgb]{0.25,0.63,0.44}{{#1}}}
    \newcommand{\FloatTok}[1]{\textcolor[rgb]{0.25,0.63,0.44}{{#1}}}
    \newcommand{\CharTok}[1]{\textcolor[rgb]{0.25,0.44,0.63}{{#1}}}
    \newcommand{\StringTok}[1]{\textcolor[rgb]{0.25,0.44,0.63}{{#1}}}
    \newcommand{\CommentTok}[1]{\textcolor[rgb]{0.38,0.63,0.69}{\textit{{#1}}}}
    \newcommand{\OtherTok}[1]{\textcolor[rgb]{0.00,0.44,0.13}{{#1}}}
    \newcommand{\AlertTok}[1]{\textcolor[rgb]{1.00,0.00,0.00}{\textbf{{#1}}}}
    \newcommand{\FunctionTok}[1]{\textcolor[rgb]{0.02,0.16,0.49}{{#1}}}
    \newcommand{\RegionMarkerTok}[1]{{#1}}
    \newcommand{\ErrorTok}[1]{\textcolor[rgb]{1.00,0.00,0.00}{\textbf{{#1}}}}
    \newcommand{\NormalTok}[1]{{#1}}
    
    % Additional commands for more recent versions of Pandoc
    \newcommand{\ConstantTok}[1]{\textcolor[rgb]{0.53,0.00,0.00}{{#1}}}
    \newcommand{\SpecialCharTok}[1]{\textcolor[rgb]{0.25,0.44,0.63}{{#1}}}
    \newcommand{\VerbatimStringTok}[1]{\textcolor[rgb]{0.25,0.44,0.63}{{#1}}}
    \newcommand{\SpecialStringTok}[1]{\textcolor[rgb]{0.73,0.40,0.53}{{#1}}}
    \newcommand{\ImportTok}[1]{{#1}}
    \newcommand{\DocumentationTok}[1]{\textcolor[rgb]{0.73,0.13,0.13}{\textit{{#1}}}}
    \newcommand{\AnnotationTok}[1]{\textcolor[rgb]{0.38,0.63,0.69}{\textbf{\textit{{#1}}}}}
    \newcommand{\CommentVarTok}[1]{\textcolor[rgb]{0.38,0.63,0.69}{\textbf{\textit{{#1}}}}}
    \newcommand{\VariableTok}[1]{\textcolor[rgb]{0.10,0.09,0.49}{{#1}}}
    \newcommand{\ControlFlowTok}[1]{\textcolor[rgb]{0.00,0.44,0.13}{\textbf{{#1}}}}
    \newcommand{\OperatorTok}[1]{\textcolor[rgb]{0.40,0.40,0.40}{{#1}}}
    \newcommand{\BuiltInTok}[1]{{#1}}
    \newcommand{\ExtensionTok}[1]{{#1}}
    \newcommand{\PreprocessorTok}[1]{\textcolor[rgb]{0.74,0.48,0.00}{{#1}}}
    \newcommand{\AttributeTok}[1]{\textcolor[rgb]{0.49,0.56,0.16}{{#1}}}
    \newcommand{\InformationTok}[1]{\textcolor[rgb]{0.38,0.63,0.69}{\textbf{\textit{{#1}}}}}
    \newcommand{\WarningTok}[1]{\textcolor[rgb]{0.38,0.63,0.69}{\textbf{\textit{{#1}}}}}
    
    
    % Define a nice break command that doesn't care if a line doesn't already
    % exist.
    \def\br{\hspace*{\fill} \\* }
    % Math Jax compatability definitions
    \def\gt{>}
    \def\lt{<}
    % Document parameters
    \title{JHU Practical Machine Learning}
    
    
    

    % Pygments definitions
    
\makeatletter
\def\PY@reset{\let\PY@it=\relax \let\PY@bf=\relax%
    \let\PY@ul=\relax \let\PY@tc=\relax%
    \let\PY@bc=\relax \let\PY@ff=\relax}
\def\PY@tok#1{\csname PY@tok@#1\endcsname}
\def\PY@toks#1+{\ifx\relax#1\empty\else%
    \PY@tok{#1}\expandafter\PY@toks\fi}
\def\PY@do#1{\PY@bc{\PY@tc{\PY@ul{%
    \PY@it{\PY@bf{\PY@ff{#1}}}}}}}
\def\PY#1#2{\PY@reset\PY@toks#1+\relax+\PY@do{#2}}

\expandafter\def\csname PY@tok@w\endcsname{\def\PY@tc##1{\textcolor[rgb]{0.73,0.73,0.73}{##1}}}
\expandafter\def\csname PY@tok@c\endcsname{\let\PY@it=\textit\def\PY@tc##1{\textcolor[rgb]{0.25,0.50,0.50}{##1}}}
\expandafter\def\csname PY@tok@cp\endcsname{\def\PY@tc##1{\textcolor[rgb]{0.74,0.48,0.00}{##1}}}
\expandafter\def\csname PY@tok@k\endcsname{\let\PY@bf=\textbf\def\PY@tc##1{\textcolor[rgb]{0.00,0.50,0.00}{##1}}}
\expandafter\def\csname PY@tok@kp\endcsname{\def\PY@tc##1{\textcolor[rgb]{0.00,0.50,0.00}{##1}}}
\expandafter\def\csname PY@tok@kt\endcsname{\def\PY@tc##1{\textcolor[rgb]{0.69,0.00,0.25}{##1}}}
\expandafter\def\csname PY@tok@o\endcsname{\def\PY@tc##1{\textcolor[rgb]{0.40,0.40,0.40}{##1}}}
\expandafter\def\csname PY@tok@ow\endcsname{\let\PY@bf=\textbf\def\PY@tc##1{\textcolor[rgb]{0.67,0.13,1.00}{##1}}}
\expandafter\def\csname PY@tok@nb\endcsname{\def\PY@tc##1{\textcolor[rgb]{0.00,0.50,0.00}{##1}}}
\expandafter\def\csname PY@tok@nf\endcsname{\def\PY@tc##1{\textcolor[rgb]{0.00,0.00,1.00}{##1}}}
\expandafter\def\csname PY@tok@nc\endcsname{\let\PY@bf=\textbf\def\PY@tc##1{\textcolor[rgb]{0.00,0.00,1.00}{##1}}}
\expandafter\def\csname PY@tok@nn\endcsname{\let\PY@bf=\textbf\def\PY@tc##1{\textcolor[rgb]{0.00,0.00,1.00}{##1}}}
\expandafter\def\csname PY@tok@ne\endcsname{\let\PY@bf=\textbf\def\PY@tc##1{\textcolor[rgb]{0.82,0.25,0.23}{##1}}}
\expandafter\def\csname PY@tok@nv\endcsname{\def\PY@tc##1{\textcolor[rgb]{0.10,0.09,0.49}{##1}}}
\expandafter\def\csname PY@tok@no\endcsname{\def\PY@tc##1{\textcolor[rgb]{0.53,0.00,0.00}{##1}}}
\expandafter\def\csname PY@tok@nl\endcsname{\def\PY@tc##1{\textcolor[rgb]{0.63,0.63,0.00}{##1}}}
\expandafter\def\csname PY@tok@ni\endcsname{\let\PY@bf=\textbf\def\PY@tc##1{\textcolor[rgb]{0.60,0.60,0.60}{##1}}}
\expandafter\def\csname PY@tok@na\endcsname{\def\PY@tc##1{\textcolor[rgb]{0.49,0.56,0.16}{##1}}}
\expandafter\def\csname PY@tok@nt\endcsname{\let\PY@bf=\textbf\def\PY@tc##1{\textcolor[rgb]{0.00,0.50,0.00}{##1}}}
\expandafter\def\csname PY@tok@nd\endcsname{\def\PY@tc##1{\textcolor[rgb]{0.67,0.13,1.00}{##1}}}
\expandafter\def\csname PY@tok@s\endcsname{\def\PY@tc##1{\textcolor[rgb]{0.73,0.13,0.13}{##1}}}
\expandafter\def\csname PY@tok@sd\endcsname{\let\PY@it=\textit\def\PY@tc##1{\textcolor[rgb]{0.73,0.13,0.13}{##1}}}
\expandafter\def\csname PY@tok@si\endcsname{\let\PY@bf=\textbf\def\PY@tc##1{\textcolor[rgb]{0.73,0.40,0.53}{##1}}}
\expandafter\def\csname PY@tok@se\endcsname{\let\PY@bf=\textbf\def\PY@tc##1{\textcolor[rgb]{0.73,0.40,0.13}{##1}}}
\expandafter\def\csname PY@tok@sr\endcsname{\def\PY@tc##1{\textcolor[rgb]{0.73,0.40,0.53}{##1}}}
\expandafter\def\csname PY@tok@ss\endcsname{\def\PY@tc##1{\textcolor[rgb]{0.10,0.09,0.49}{##1}}}
\expandafter\def\csname PY@tok@sx\endcsname{\def\PY@tc##1{\textcolor[rgb]{0.00,0.50,0.00}{##1}}}
\expandafter\def\csname PY@tok@m\endcsname{\def\PY@tc##1{\textcolor[rgb]{0.40,0.40,0.40}{##1}}}
\expandafter\def\csname PY@tok@gh\endcsname{\let\PY@bf=\textbf\def\PY@tc##1{\textcolor[rgb]{0.00,0.00,0.50}{##1}}}
\expandafter\def\csname PY@tok@gu\endcsname{\let\PY@bf=\textbf\def\PY@tc##1{\textcolor[rgb]{0.50,0.00,0.50}{##1}}}
\expandafter\def\csname PY@tok@gd\endcsname{\def\PY@tc##1{\textcolor[rgb]{0.63,0.00,0.00}{##1}}}
\expandafter\def\csname PY@tok@gi\endcsname{\def\PY@tc##1{\textcolor[rgb]{0.00,0.63,0.00}{##1}}}
\expandafter\def\csname PY@tok@gr\endcsname{\def\PY@tc##1{\textcolor[rgb]{1.00,0.00,0.00}{##1}}}
\expandafter\def\csname PY@tok@ge\endcsname{\let\PY@it=\textit}
\expandafter\def\csname PY@tok@gs\endcsname{\let\PY@bf=\textbf}
\expandafter\def\csname PY@tok@gp\endcsname{\let\PY@bf=\textbf\def\PY@tc##1{\textcolor[rgb]{0.00,0.00,0.50}{##1}}}
\expandafter\def\csname PY@tok@go\endcsname{\def\PY@tc##1{\textcolor[rgb]{0.53,0.53,0.53}{##1}}}
\expandafter\def\csname PY@tok@gt\endcsname{\def\PY@tc##1{\textcolor[rgb]{0.00,0.27,0.87}{##1}}}
\expandafter\def\csname PY@tok@err\endcsname{\def\PY@bc##1{\setlength{\fboxsep}{0pt}\fcolorbox[rgb]{1.00,0.00,0.00}{1,1,1}{\strut ##1}}}
\expandafter\def\csname PY@tok@kc\endcsname{\let\PY@bf=\textbf\def\PY@tc##1{\textcolor[rgb]{0.00,0.50,0.00}{##1}}}
\expandafter\def\csname PY@tok@kd\endcsname{\let\PY@bf=\textbf\def\PY@tc##1{\textcolor[rgb]{0.00,0.50,0.00}{##1}}}
\expandafter\def\csname PY@tok@kn\endcsname{\let\PY@bf=\textbf\def\PY@tc##1{\textcolor[rgb]{0.00,0.50,0.00}{##1}}}
\expandafter\def\csname PY@tok@kr\endcsname{\let\PY@bf=\textbf\def\PY@tc##1{\textcolor[rgb]{0.00,0.50,0.00}{##1}}}
\expandafter\def\csname PY@tok@bp\endcsname{\def\PY@tc##1{\textcolor[rgb]{0.00,0.50,0.00}{##1}}}
\expandafter\def\csname PY@tok@fm\endcsname{\def\PY@tc##1{\textcolor[rgb]{0.00,0.00,1.00}{##1}}}
\expandafter\def\csname PY@tok@vc\endcsname{\def\PY@tc##1{\textcolor[rgb]{0.10,0.09,0.49}{##1}}}
\expandafter\def\csname PY@tok@vg\endcsname{\def\PY@tc##1{\textcolor[rgb]{0.10,0.09,0.49}{##1}}}
\expandafter\def\csname PY@tok@vi\endcsname{\def\PY@tc##1{\textcolor[rgb]{0.10,0.09,0.49}{##1}}}
\expandafter\def\csname PY@tok@vm\endcsname{\def\PY@tc##1{\textcolor[rgb]{0.10,0.09,0.49}{##1}}}
\expandafter\def\csname PY@tok@sa\endcsname{\def\PY@tc##1{\textcolor[rgb]{0.73,0.13,0.13}{##1}}}
\expandafter\def\csname PY@tok@sb\endcsname{\def\PY@tc##1{\textcolor[rgb]{0.73,0.13,0.13}{##1}}}
\expandafter\def\csname PY@tok@sc\endcsname{\def\PY@tc##1{\textcolor[rgb]{0.73,0.13,0.13}{##1}}}
\expandafter\def\csname PY@tok@dl\endcsname{\def\PY@tc##1{\textcolor[rgb]{0.73,0.13,0.13}{##1}}}
\expandafter\def\csname PY@tok@s2\endcsname{\def\PY@tc##1{\textcolor[rgb]{0.73,0.13,0.13}{##1}}}
\expandafter\def\csname PY@tok@sh\endcsname{\def\PY@tc##1{\textcolor[rgb]{0.73,0.13,0.13}{##1}}}
\expandafter\def\csname PY@tok@s1\endcsname{\def\PY@tc##1{\textcolor[rgb]{0.73,0.13,0.13}{##1}}}
\expandafter\def\csname PY@tok@mb\endcsname{\def\PY@tc##1{\textcolor[rgb]{0.40,0.40,0.40}{##1}}}
\expandafter\def\csname PY@tok@mf\endcsname{\def\PY@tc##1{\textcolor[rgb]{0.40,0.40,0.40}{##1}}}
\expandafter\def\csname PY@tok@mh\endcsname{\def\PY@tc##1{\textcolor[rgb]{0.40,0.40,0.40}{##1}}}
\expandafter\def\csname PY@tok@mi\endcsname{\def\PY@tc##1{\textcolor[rgb]{0.40,0.40,0.40}{##1}}}
\expandafter\def\csname PY@tok@il\endcsname{\def\PY@tc##1{\textcolor[rgb]{0.40,0.40,0.40}{##1}}}
\expandafter\def\csname PY@tok@mo\endcsname{\def\PY@tc##1{\textcolor[rgb]{0.40,0.40,0.40}{##1}}}
\expandafter\def\csname PY@tok@ch\endcsname{\let\PY@it=\textit\def\PY@tc##1{\textcolor[rgb]{0.25,0.50,0.50}{##1}}}
\expandafter\def\csname PY@tok@cm\endcsname{\let\PY@it=\textit\def\PY@tc##1{\textcolor[rgb]{0.25,0.50,0.50}{##1}}}
\expandafter\def\csname PY@tok@cpf\endcsname{\let\PY@it=\textit\def\PY@tc##1{\textcolor[rgb]{0.25,0.50,0.50}{##1}}}
\expandafter\def\csname PY@tok@c1\endcsname{\let\PY@it=\textit\def\PY@tc##1{\textcolor[rgb]{0.25,0.50,0.50}{##1}}}
\expandafter\def\csname PY@tok@cs\endcsname{\let\PY@it=\textit\def\PY@tc##1{\textcolor[rgb]{0.25,0.50,0.50}{##1}}}

\def\PYZbs{\char`\\}
\def\PYZus{\char`\_}
\def\PYZob{\char`\{}
\def\PYZcb{\char`\}}
\def\PYZca{\char`\^}
\def\PYZam{\char`\&}
\def\PYZlt{\char`\<}
\def\PYZgt{\char`\>}
\def\PYZsh{\char`\#}
\def\PYZpc{\char`\%}
\def\PYZdl{\char`\$}
\def\PYZhy{\char`\-}
\def\PYZsq{\char`\'}
\def\PYZdq{\char`\"}
\def\PYZti{\char`\~}
% for compatibility with earlier versions
\def\PYZat{@}
\def\PYZlb{[}
\def\PYZrb{]}
\makeatother


    % Exact colors from NB
    \definecolor{incolor}{rgb}{0.0, 0.0, 0.5}
    \definecolor{outcolor}{rgb}{0.545, 0.0, 0.0}



    
    % Prevent overflowing lines due to hard-to-break entities
    \sloppy 
    % Setup hyperref package
    \hypersetup{
      breaklinks=true,  % so long urls are correctly broken across lines
      colorlinks=true,
      urlcolor=urlcolor,
      linkcolor=linkcolor,
      citecolor=citecolor,
      }
    % Slightly bigger margins than the latex defaults
    
    \geometry{verbose,tmargin=1in,bmargin=1in,lmargin=1in,rmargin=1in}
    
    

    \begin{document}
    
    
    \maketitle
    
    

    
    \section{JHU Practical Machine
Learning}\label{jhu-practical-machine-learning}

Course URL:
https://www.coursera.org/learn/practical-machine-learning/home/welcome

    \subsubsection{Lesson 1 - What is
Prediction?}\label{lesson-1---what-is-prediction}

Basic ML Workflow

question -\textgreater{} input data -\textgreater{} features
-\textgreater{} algorithm -\textgreater{} parameters -\textgreater{}
evaluation \#\#\#\#\#\# SPAM Example

    \begin{Verbatim}[commandchars=\\\{\}]
{\color{incolor}In [{\color{incolor}57}]:} \PY{c+c1}{\PYZsh{} 1. Load and examine dataset}
         \PY{c+c1}{\PYZsh{} install.packages(\PYZdq{}kernlab\PYZdq{})}
         \PY{k+kn}{library}\PY{p}{(}kernlab\PY{p}{)}
         data\PY{p}{(}spam\PY{p}{)}
         \PY{k+kp}{paste}\PY{p}{(}\PY{l+s}{\PYZsq{}}\PY{l+s}{Columns: \PYZsq{}}\PY{p}{,} \PY{k+kp}{length}\PY{p}{(}\PY{k+kp}{names}\PY{p}{(}spam\PY{p}{)}\PY{p}{)}\PY{p}{)}
         \PY{k+kp}{head}\PY{p}{(}spam\PY{p}{)}
\end{Verbatim}


    'Columns:  58'

    
    \begin{tabular}{r|llllllllllllllllllllllllllllllllllllllllllllllllllllllllll}
 make & address & all & num3d & our & over & remove & internet & order & mail & ⋯ & charSemicolon & charRoundbracket & charSquarebracket & charExclamation & charDollar & charHash & capitalAve & capitalLong & capitalTotal & type\\
\hline
	 0.00  & 0.64  & 0.64  & 0     & 0.32  & 0.00  & 0.00  & 0.00  & 0.00  & 0.00  & ⋯     & 0.00  & 0.000 & 0     & 0.778 & 0.000 & 0.000 & 3.756 &  61   &  278  & spam \\
	 0.21  & 0.28  & 0.50  & 0     & 0.14  & 0.28  & 0.21  & 0.07  & 0.00  & 0.94  & ⋯     & 0.00  & 0.132 & 0     & 0.372 & 0.180 & 0.048 & 5.114 & 101   & 1028  & spam \\
	 0.06  & 0.00  & 0.71  & 0     & 1.23  & 0.19  & 0.19  & 0.12  & 0.64  & 0.25  & ⋯     & 0.01  & 0.143 & 0     & 0.276 & 0.184 & 0.010 & 9.821 & 485   & 2259  & spam \\
	 0.00  & 0.00  & 0.00  & 0     & 0.63  & 0.00  & 0.31  & 0.63  & 0.31  & 0.63  & ⋯     & 0.00  & 0.137 & 0     & 0.137 & 0.000 & 0.000 & 3.537 &  40   &  191  & spam \\
	 0.00  & 0.00  & 0.00  & 0     & 0.63  & 0.00  & 0.31  & 0.63  & 0.31  & 0.63  & ⋯     & 0.00  & 0.135 & 0     & 0.135 & 0.000 & 0.000 & 3.537 &  40   &  191  & spam \\
	 0.00  & 0.00  & 0.00  & 0     & 1.85  & 0.00  & 0.00  & 1.85  & 0.00  & 0.00  & ⋯     & 0.00  & 0.223 & 0     & 0.000 & 0.000 & 0.000 & 3.000 &  15   &   54  & spam \\
\end{tabular}


    
    \begin{Verbatim}[commandchars=\\\{\}]
{\color{incolor}In [{\color{incolor}60}]:} \PY{c+c1}{\PYZsh{} This dataset is an example of what could be used as training data for a classification problem.}
         \PY{c+c1}{\PYZsh{} Interestingly, all variables except for 1 are features. The remaining one is the labeled classification.}
         \PY{o}{?}spam
\end{Verbatim}


    \begin{Verbatim}[commandchars=\\\{\}]
{\color{incolor}In [{\color{incolor}58}]:} \PY{k+kp}{names}\PY{p}{(}spam\PY{p}{)}
\end{Verbatim}


    \begin{enumerate*}
\item 'make'
\item 'address'
\item 'all'
\item 'num3d'
\item 'our'
\item 'over'
\item 'remove'
\item 'internet'
\item 'order'
\item 'mail'
\item 'receive'
\item 'will'
\item 'people'
\item 'report'
\item 'addresses'
\item 'free'
\item 'business'
\item 'email'
\item 'you'
\item 'credit'
\item 'your'
\item 'font'
\item 'num000'
\item 'money'
\item 'hp'
\item 'hpl'
\item 'george'
\item 'num650'
\item 'lab'
\item 'labs'
\item 'telnet'
\item 'num857'
\item 'data'
\item 'num415'
\item 'num85'
\item 'technology'
\item 'num1999'
\item 'parts'
\item 'pm'
\item 'direct'
\item 'cs'
\item 'meeting'
\item 'original'
\item 'project'
\item 're'
\item 'edu'
\item 'table'
\item 'conference'
\item 'charSemicolon'
\item 'charRoundbracket'
\item 'charSquarebracket'
\item 'charExclamation'
\item 'charDollar'
\item 'charHash'
\item 'capitalAve'
\item 'capitalLong'
\item 'capitalTotal'
\item 'type'
\end{enumerate*}


    
    \begin{Verbatim}[commandchars=\\\{\}]
{\color{incolor}In [{\color{incolor}50}]:} str\PY{p}{(}spam\PY{p}{)}
\end{Verbatim}


    \begin{Verbatim}[commandchars=\\\{\}]
'data.frame':	4601 obs. of  58 variables:
 \$ make             : num  0 0.21 0.06 0 0 0 0 0 0.15 0.06 {\ldots}
 \$ address          : num  0.64 0.28 0 0 0 0 0 0 0 0.12 {\ldots}
 \$ all              : num  0.64 0.5 0.71 0 0 0 0 0 0.46 0.77 {\ldots}
 \$ num3d            : num  0 0 0 0 0 0 0 0 0 0 {\ldots}
 \$ our              : num  0.32 0.14 1.23 0.63 0.63 1.85 1.92 1.88 0.61 0.19 {\ldots}
 \$ over             : num  0 0.28 0.19 0 0 0 0 0 0 0.32 {\ldots}
 \$ remove           : num  0 0.21 0.19 0.31 0.31 0 0 0 0.3 0.38 {\ldots}
 \$ internet         : num  0 0.07 0.12 0.63 0.63 1.85 0 1.88 0 0 {\ldots}
 \$ order            : num  0 0 0.64 0.31 0.31 0 0 0 0.92 0.06 {\ldots}
 \$ mail             : num  0 0.94 0.25 0.63 0.63 0 0.64 0 0.76 0 {\ldots}
 \$ receive          : num  0 0.21 0.38 0.31 0.31 0 0.96 0 0.76 0 {\ldots}
 \$ will             : num  0.64 0.79 0.45 0.31 0.31 0 1.28 0 0.92 0.64 {\ldots}
 \$ people           : num  0 0.65 0.12 0.31 0.31 0 0 0 0 0.25 {\ldots}
 \$ report           : num  0 0.21 0 0 0 0 0 0 0 0 {\ldots}
 \$ addresses        : num  0 0.14 1.75 0 0 0 0 0 0 0.12 {\ldots}
 \$ free             : num  0.32 0.14 0.06 0.31 0.31 0 0.96 0 0 0 {\ldots}
 \$ business         : num  0 0.07 0.06 0 0 0 0 0 0 0 {\ldots}
 \$ email            : num  1.29 0.28 1.03 0 0 0 0.32 0 0.15 0.12 {\ldots}
 \$ you              : num  1.93 3.47 1.36 3.18 3.18 0 3.85 0 1.23 1.67 {\ldots}
 \$ credit           : num  0 0 0.32 0 0 0 0 0 3.53 0.06 {\ldots}
 \$ your             : num  0.96 1.59 0.51 0.31 0.31 0 0.64 0 2 0.71 {\ldots}
 \$ font             : num  0 0 0 0 0 0 0 0 0 0 {\ldots}
 \$ num000           : num  0 0.43 1.16 0 0 0 0 0 0 0.19 {\ldots}
 \$ money            : num  0 0.43 0.06 0 0 0 0 0 0.15 0 {\ldots}
 \$ hp               : num  0 0 0 0 0 0 0 0 0 0 {\ldots}
 \$ hpl              : num  0 0 0 0 0 0 0 0 0 0 {\ldots}
 \$ george           : num  0 0 0 0 0 0 0 0 0 0 {\ldots}
 \$ num650           : num  0 0 0 0 0 0 0 0 0 0 {\ldots}
 \$ lab              : num  0 0 0 0 0 0 0 0 0 0 {\ldots}
 \$ labs             : num  0 0 0 0 0 0 0 0 0 0 {\ldots}
 \$ telnet           : num  0 0 0 0 0 0 0 0 0 0 {\ldots}
 \$ num857           : num  0 0 0 0 0 0 0 0 0 0 {\ldots}
 \$ data             : num  0 0 0 0 0 0 0 0 0.15 0 {\ldots}
 \$ num415           : num  0 0 0 0 0 0 0 0 0 0 {\ldots}
 \$ num85            : num  0 0 0 0 0 0 0 0 0 0 {\ldots}
 \$ technology       : num  0 0 0 0 0 0 0 0 0 0 {\ldots}
 \$ num1999          : num  0 0.07 0 0 0 0 0 0 0 0 {\ldots}
 \$ parts            : num  0 0 0 0 0 0 0 0 0 0 {\ldots}
 \$ pm               : num  0 0 0 0 0 0 0 0 0 0 {\ldots}
 \$ direct           : num  0 0 0.06 0 0 0 0 0 0 0 {\ldots}
 \$ cs               : num  0 0 0 0 0 0 0 0 0 0 {\ldots}
 \$ meeting          : num  0 0 0 0 0 0 0 0 0 0 {\ldots}
 \$ original         : num  0 0 0.12 0 0 0 0 0 0.3 0 {\ldots}
 \$ project          : num  0 0 0 0 0 0 0 0 0 0.06 {\ldots}
 \$ re               : num  0 0 0.06 0 0 0 0 0 0 0 {\ldots}
 \$ edu              : num  0 0 0.06 0 0 0 0 0 0 0 {\ldots}
 \$ table            : num  0 0 0 0 0 0 0 0 0 0 {\ldots}
 \$ conference       : num  0 0 0 0 0 0 0 0 0 0 {\ldots}
 \$ charSemicolon    : num  0 0 0.01 0 0 0 0 0 0 0.04 {\ldots}
 \$ charRoundbracket : num  0 0.132 0.143 0.137 0.135 0.223 0.054 0.206 0.271 0.03 {\ldots}
 \$ charSquarebracket: num  0 0 0 0 0 0 0 0 0 0 {\ldots}
 \$ charExclamation  : num  0.778 0.372 0.276 0.137 0.135 0 0.164 0 0.181 0.244 {\ldots}
 \$ charDollar       : num  0 0.18 0.184 0 0 0 0.054 0 0.203 0.081 {\ldots}
 \$ charHash         : num  0 0.048 0.01 0 0 0 0 0 0.022 0 {\ldots}
 \$ capitalAve       : num  3.76 5.11 9.82 3.54 3.54 {\ldots}
 \$ capitalLong      : num  61 101 485 40 40 15 4 11 445 43 {\ldots}
 \$ capitalTotal     : num  278 1028 2259 191 191 {\ldots}
 \$ type             : Factor w/ 2 levels "nonspam","spam": 2 2 2 2 2 2 2 2 2 2 {\ldots}

    \end{Verbatim}

    \begin{Verbatim}[commandchars=\\\{\}]
{\color{incolor}In [{\color{incolor}41}]:} \PY{k+kt}{c}\PY{p}{(}\PY{k+kp}{min}\PY{p}{(}spam\PY{o}{\PYZdl{}}your\PY{p}{)}\PY{p}{,} \PY{k+kp}{max}\PY{p}{(}spam\PY{o}{\PYZdl{}}your\PY{p}{)}\PY{p}{)}
\end{Verbatim}


    \begin{enumerate*}
\item 0
\item 11.11
\end{enumerate*}


    
    \begin{Verbatim}[commandchars=\\\{\}]
{\color{incolor}In [{\color{incolor}42}]:} density\PY{p}{(}spam\PY{o}{\PYZdl{}}your\PY{p}{[}spam\PY{o}{\PYZdl{}}type\PY{o}{==}\PY{l+s}{\PYZdq{}}\PY{l+s}{nonspam\PYZdq{}}\PY{p}{]}\PY{p}{)}
\end{Verbatim}


    
    \begin{verbatim}

Call:
	density.default(x = spam$your[spam$type == "nonspam"])

Data: spam$your[spam$type == "nonspam"] (2788 obs.);	Bandwidth 'bw' = 0.06322

       x                 y           
 Min.   :-0.1897   Min.   :0.000000  
 1st Qu.: 2.5827   1st Qu.:0.000459  
 Median : 5.3550   Median :0.003009  
 Mean   : 5.3550   Mean   :0.089983  
 3rd Qu.: 8.1273   3rd Qu.:0.030547  
 Max.   :10.8997   Max.   :4.061939  
    \end{verbatim}

    
    \begin{Verbatim}[commandchars=\\\{\}]
{\color{incolor}In [{\color{incolor}63}]:} \PY{c+c1}{\PYZsh{} Y axis = Density of being spam (blue) or not spam (red)}
         plot\PY{p}{(}density\PY{p}{(}spam\PY{o}{\PYZdl{}}your\PY{p}{[}spam\PY{o}{\PYZdl{}}type\PY{o}{==}\PY{l+s}{\PYZdq{}}\PY{l+s}{nonspam\PYZdq{}}\PY{p}{]}\PY{p}{)}\PY{p}{,} col\PY{o}{=}\PY{l+s}{\PYZdq{}}\PY{l+s}{blue\PYZdq{}}\PY{p}{,} main\PY{o}{=}\PY{l+s}{\PYZdq{}}\PY{l+s}{\PYZdq{}}\PY{p}{,} xlab\PY{o}{=}\PY{l+s}{\PYZdq{}}\PY{l+s}{Frequency of \PYZsq{}your\PYZsq{}\PYZdq{}}\PY{p}{)}
         lines\PY{p}{(}density\PY{p}{(}spam\PY{o}{\PYZdl{}}your\PY{p}{[}spam\PY{o}{\PYZdl{}}type\PY{o}{==}\PY{l+s}{\PYZdq{}}\PY{l+s}{spam\PYZdq{}}\PY{p}{]}\PY{p}{)}\PY{p}{,} col\PY{o}{=}\PY{l+s}{\PYZdq{}}\PY{l+s}{red\PYZdq{}}\PY{p}{)}
         \PY{c+c1}{\PYZsh{} Perhaps above a 0.5 \PYZdq{}uses of \PYZsq{}your\PYZsq{}\PYZdq{}/e\PYZhy{}mail frequency cutoff is good for classifying as spam? Of course though, there\PYZsq{}s no such thing as \PYZdq{}0.5\PYZdq{} uses of a word in an e\PYZhy{}mail? Unless perhaps the dataset means multiple e\PYZhy{}mails from the same person.}
         abline\PY{p}{(}v\PY{o}{=}\PY{l+m}{0.5}\PY{p}{,} col\PY{o}{=}\PY{l+s}{\PYZdq{}}\PY{l+s}{black\PYZdq{}}\PY{p}{)}
\end{Verbatim}


    \begin{center}
    \adjustimage{max size={0.9\linewidth}{0.9\paperheight}}{output_8_0.png}
    \end{center}
    { \hspace*{\fill} \\}
    
    \begin{Verbatim}[commandchars=\\\{\}]
{\color{incolor}In [{\color{incolor}66}]:} \PY{k+kp}{summary}\PY{p}{(}spam\PY{o}{\PYZdl{}}type\PY{p}{)}
\end{Verbatim}


    \begin{description*}
\item[nonspam] 2788
\item[spam] 1813
\end{description*}


    
    Evaluating Accuracy of a Binary Classifier

In this case, I call it 'binary' because there is only spam/nonspam.
Accuracy evaluation can be thought of as something like "the percentage
of spam that was classified correctly as spam, plus the same for
nonspam".

The same number could be reached by starting with 1 and subtracing the
misclassified spam and the misclassified non-spam percentages.

This is apparently an "optimistic" estimate of the overall error rate,
which I will learn about later in the course.

    \begin{Verbatim}[commandchars=\\\{\}]
{\color{incolor}In [{\color{incolor}86}]:} prediction \PY{o}{\PYZlt{}\PYZhy{}} \PY{k+kp}{ifelse}\PY{p}{(}spam\PY{o}{\PYZdl{}}your \PY{o}{\PYZgt{}} \PY{l+m}{0.5}\PY{p}{,} \PY{l+s}{\PYZsq{}}\PY{l+s}{spam\PYZsq{}}\PY{p}{,} \PY{l+s}{\PYZsq{}}\PY{l+s}{nonspam\PYZsq{}}\PY{p}{)}
         \PY{k+kp}{paste}\PY{p}{(}\PY{l+s}{\PYZsq{}}\PY{l+s}{Predicted Spam: \PYZsq{}}\PY{p}{,} \PY{k+kp}{length}\PY{p}{(}prediction\PY{p}{[}prediction\PY{o}{==}\PY{l+s}{\PYZsq{}}\PY{l+s}{spam\PYZsq{}}\PY{p}{]}\PY{p}{)}\PY{p}{,} \PY{l+s}{\PYZsq{}}\PY{l+s}{ / \PYZsq{}}\PY{p}{,} \PY{l+s}{\PYZsq{}}\PY{l+s}{Predicted Non\PYZhy{}spam: \PYZsq{}}\PY{p}{,} \PY{k+kp}{length}\PY{p}{(}prediction\PY{p}{[}prediction\PY{o}{==}\PY{l+s}{\PYZsq{}}\PY{l+s}{nonspam\PYZsq{}}\PY{p}{]}\PY{p}{)}\PY{p}{)}
         prediction\PYZus{}table \PY{o}{\PYZlt{}\PYZhy{}} \PY{k+kp}{table}\PY{p}{(}prediction\PY{p}{,} spam\PY{o}{\PYZdl{}}type\PY{p}{)}\PY{o}{/}\PY{k+kp}{length}\PY{p}{(}spam\PY{o}{\PYZdl{}}type\PY{p}{)}
         prediction\PYZus{}table
         \PY{k+kp}{paste}\PY{p}{(}\PY{l+s}{\PYZsq{}}\PY{l+s}{Accuracy: \PYZsq{}}\PY{p}{,} prediction\PYZus{}table\PY{p}{[}\PY{l+s}{\PYZsq{}}\PY{l+s}{spam\PYZsq{}}\PY{p}{,} \PY{l+s}{\PYZsq{}}\PY{l+s}{spam\PYZsq{}}\PY{p}{]} \PY{o}{+} prediction\PYZus{}table\PY{p}{[}\PY{l+s}{\PYZsq{}}\PY{l+s}{nonspam\PYZsq{}}\PY{p}{,} \PY{l+s}{\PYZsq{}}\PY{l+s}{nonspam\PYZsq{}}\PY{p}{]}\PY{p}{)}
\end{Verbatim}


    'Predicted Spam:  2021  /  Predicted Non-spam:  2580'

    
    
    \begin{verbatim}
          
prediction   nonspam      spam
   nonspam 0.4590306 0.1017170
   spam    0.1469246 0.2923278
    \end{verbatim}

    
    'Accuracy:  0.75135840034775'

    

    % Add a bibliography block to the postdoc
    
    
    
    \end{document}
